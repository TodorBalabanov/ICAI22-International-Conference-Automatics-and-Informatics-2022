\documentclass[conference]{IEEEtran}
\IEEEoverridecommandlockouts

%\usepackage{cite}
\usepackage{amsmath,amssymb,amsfonts}
\usepackage{algorithmic}
\usepackage{graphicx}
\usepackage{natbib}
\usepackage{textcomp}
\usepackage{xcolor}
\def\BibTeX{{\rm B\kern-.05em{\sc i\kern-.025em b}\kern-.08em T\kern-.1667em\lower.7ex\hbox{E}\kern-.125emX}}

\begin{document}
\bibliographystyle{IEEEtran}

\title{An Efficiency of Third Party Genetic Algorithms Software Libraries in Mobile Distributed Computing for Financial Time Series Forecasting\\
{\footnotesize \textsuperscript{}}
\thanks{This research is funded by Velbazhd Software LLC. It is partially supported by the Ministry of Education and Science of the Republic Bulgaria under the National Science Program “Intelligentanimal husbandry”, grant agreement No. D01-62/18.03.2021/, the National Research Programme “Young scientists and postdoctoral students” approved by DCM No. 577/17.08.2018, and the Bulgarian National Science Fund by the project “Mathematical models, methods and algorithms for solving hard optimization problems to achieve high security in communications and better economic sustainability”, KP-06-N52/7/19-11-2021.}
}

\author{\IEEEauthorblockN{1\textsuperscript{st} Gergana Mateeva}
\IEEEauthorblockA{\textit{Institute of Information and Communication Technologies} \\
\textit{Bulgarian Academy of Sciences}\\
Sofia, Bulgaria \\
gergana.mateeva@iict.bas.bg}
\and
\IEEEauthorblockN{2\textsuperscript{nd} Dimitar Parvanov}
\IEEEauthorblockA{\textit{Institute of Information and Communication Technologies} \\
\textit{Bulgarian Academy of Sciences}\\
Sofia, Bulgaria \\
dimitar.parvanov@iict.bas.bg}
\and
\IEEEauthorblockN{3\textsuperscript{rd} Ioan Dimitrov}
\IEEEauthorblockA{\textit{Faculty of Electronic Engineering and Technology} \\
\textit{Technical University of Sofia}\\
Sofia, Bulgaria \\
joancdimitrov@tu-sofia.bg}
\and
\IEEEauthorblockN{4\textsuperscript{th} Iliyan Iliev}
\IEEEauthorblockA{\textit{Institute of Information and Communication Technologies} \\
\textit{Bulgarian Academy of Sciences}\\
Sofia, Bulgaria \\
iliyan.iliev@iict.bas.bg}
\and
\IEEEauthorblockN{5\textsuperscript{th} Todor Balabanov}
\IEEEauthorblockA{\textit{Institute of Information and Communication Technologies} \\
\textit{Bulgarian Academy of Sciences}\\
Sofia, Bulgaria \\
todor.balabanov@iict.bas.bg}
}

\maketitle

\begin{abstract}
Genetic Algorithms are very well-known optimization metaheuristics. They are very well presented in mathematical applications like Matlab, R, and others. Such specific implementations are not proper for industrial software development. Because of its popularity, Genetic Algorithms have become implemented as third-party software libraries. The popularity of Android OS with its capabilities for running Java source code attracted the usage of external software libraries for achieving mobile distributed computing tasks. Training of Artificial Neural Networks and Curve Fitting by the usage of Genetic Algorithms brought Financial Time Series Forecasting to the mobile world. In this study, two of the most popular Genetic Algorithm software libraries are compared in order to be used in mobile distributed computing application.
\end{abstract}

\begin{IEEEkeywords}
genetic algorithms, mobile distributed computing, software libraries
\end{IEEEkeywords}

\section{Introduction}

In the face of mobile distributed computing \cite{Bibi-2021-a}, state of the art in soft computing \cite{Angelova-2009-a} received a valuable extension. Additional acceleration of mobile device usage adoption has been given by Covid-19 pandemic \cite{Petrov-2021-a}. Wireless sensor networks \cite{Alexandrov-2016-a} in combination with smart routing \cite{Tashev-2019-a} and IoT \cite{Dineva-2019-a} were a valuable idea source for mobile distributed computing implementations. The usage of heterogeneous devices \cite{Weinbub-2012-a}  appears to be very efficient for slipping calculations in complex combinatorial problems \cite{Borissova-2015-a}. When calculating devices are in a volunteer scheme the control over them is unknown \cite{Balabanov-2020-a}, which leads to a number of cybersecurity issues \cite{Dimitrov-2021-a}. 

In this study, a mobile distributed computing software \cite{Balabanov-2022-a} has been extended. The extension comes to the machine learning algorithms applied in financial time series forecasting. Two of the most popular software libraries for genetic algorithms are included (MOEA Framework \cite{Huo-2018-a} and Jenetics \cite{Aalam-2022-a}). 

The paper is organized as follows: Section 2 introduces mobile distributed computing and its specifics; Section 2 presents financial time series forecasting and some of the related challenges; Section 3 reveals some experiments and results; Section 4 concludes and provides some guidance for further research.

\section{Mobile Distributed Computing}

\subsection{}

\section{Time Series Forecasting}

\subsection{}

\section{Experiments \& Results}

\subsection{}

\section{Conclusion}

\section*{Acknowledgment}

This research is funded by Velbazhd Software LLC. It is partially supported by the Ministry of Education and Science of the Republic Bulgaria under the National Science Program “Intelligentanimal husbandry”, grant agreement No. D01-62/18.03.2021/, the National Research Programme “Young scientists and postdoctoral students” approved by DCM No. 577/17.08.2018, and the Bulgarian National Science Fund by the project “Mathematical models, methods and algorithms for solving hard optimization problems to achieve high security in communications and better economic sustainability”, KP-06-N52/7/19-11-2021.

\bibliography{references}

\end{document}
